\newcommand{\orgmode}{\texttt{org-mode}}
\newcommand{\Makefile}{\mintinline{shell}{Makefile}}
\newcommand{\context}{\mintinline{shell}{context}}
\newcommand{\CC}{C\nolinebreak\hspace{-.05em}\raisebox{.4ex}{\tiny\bf +}\nolinebreak\hspace{-.10em}\raisebox{.4ex}{\tiny\bf +}}
\def\CC{{C\nolinebreak[4]\hspace{-.05em}\raisebox{.4ex}{\tiny\bf ++}}}

\section{Introduction}

\subsection{Summary of report D1.1}

\Ac{QMCkl} aims at providing a high-performance
implementation of the main kernels of \ac{QMC} methods. 
\ac{WP}~1 focuses on defining the \ac{API}, the tests,
and a \emph{pedagogical} presentation of the main algorithms.
In \ac{WP}~3, the \ac{HPC} experts use this repository as a reference to re-write
optimized versions of the functions proposed in this pedagogical version of the library.

To maximize portability to unknown architectures, we have chosen to provide a \ac{API} that
is compatible with the C programming language. The foundations of the library are written in C, but Fortran
was chosen to express the kernels as it is more convenient than C to express linear algebra.
The dependencies on external software are kept minimal to facilitate the installation, and the
BSD 3-clause license was chosen to facilitate the adoption of the library by industrial 
collaborators.

As this version of the library is intended to be mainly used as documentation, 
we chose to use literate programming\cite{knuth_1992} in {\orgmode}
format.~\cite{schulte_2012,orgmode} The documentation
is first written, and the code illustrates the documentation. From the same source files,
the documentation is generated both in text and HTML formats, as well as the C code of the library.
This practice helps documentation to always be consistent with the code.

The documentation of the current status of the library is available
at \url{https://trex-coe.github.io/qmckl}, and the source code is
available on the GitHub repository at \url{https://github.com/trex-coe/qmckl}.

\subsection{QMCkl's objective}

\newcommand{\Nelec}{N_{\text{elec}}}
\newcommand{\Nelecup}{N_{\text{elec}}^\uparrow}
\newcommand{\Nelecdn}{N_{\text{elec}}^\downarrow}
\newcommand{\br}{\mathbf{r}_1,\dots,\mathbf{r}_{\Nelec}}
\newcommand{\brup}{\mathbf{r}_1,\dots,\mathbf{r}_{\Nelec^\uparrow}}
\newcommand{\brdn}{\mathbf{r}_{\Nelec^\uparrow+1},\dots,\mathbf{r}_{\Nelec}}

The most common form of wave function used in \ac{QMC} simulations is
expressed as a linear combination of Slater determinants $D_I$ multiplied by
a Jastrow correlation factor $\exp(J)$:
\begin{equation}
  \Psi(\br) = \left( \sum_I c_I\, D_I(\br) \right) \exp \left( J(\br) \right).
\end{equation}

The Jastrow factor is a positive function of the electron-electron,
electron-nucleus, and electron-electron-nucleus distances. 

The Slater determinants are expressed as products of $\uparrow$-spin
$\downarrow$-spin determinants:
\begin{eqnarray}
D_I(\br) = D_i^\uparrow(\brup)\, D_j^\downarrow(\brdn),
\end{eqnarray}
where $\Nelec = \Nelecup + \Nelecdn$ is the number of
electrons, the sum of the numbers of up- and down-spin electrons.

The determinants are obtained from the Slater matrix, built on the basis of \acp{MO},
\begin{equation}
  D_i^\uparrow(\brup) = det(S_i^\uparrow) = \left|
    \begin{array}{ccc}
      \phi_{i(1)}(\mathbf{r}_1) & \dots & \phi_{i(\Nelecup)}(\mathbf{r}_1) \\
        \vdots & \ddots & \vdots \\
      \phi_{i(1)}(\mathbf{r}_{\Nelecup}) & \dots & \phi_{i(\Nelecup)}(\mathbf{r}_{\Nelecup}) 
    \end{array}
    \right|
\end{equation}
and each Slater matrix differs from the other ones by the indices $\{
i(1), \dots, i(\Nelecup) \}$ of the \acp{MO} composing it.

The \acp{MO} $\phi$ are linear combinations of \acp{AO} $\chi$:
\begin{equation}
\phi_j(\mathbf{r}) = \sum_k C_{kj} \chi_k(\mathbf{r}),
\end{equation}
each \ac{AO} being a function centered on a nucleus with a radial and an angular part.
The angular part is whether expressed as spherical harmonics $Y_{lm}$ or as a polynomial, and
the angular part is generally a linear combination of Gaussian functions:
\begin{equation}
\chi_k(\mathbf{r}) = (x-X_A)^a (y-Y_A)^b (z-Z_A)^c \sum_l d_l\, \exp \left(
    -\gamma_{kl} |\mathbf{r}-\mathbf{R}_A|^2 \right)
\end{equation}
where $\mathbf{r} = (x,y,z)$ is an electron coordinate and 
$\mathbf{R_A} = (X_A,Y_A,Z_A)$ is the coordinate of nucleus $A$.

At each step of the \ac{QMC} simulation, the wave function must be
evaluated at the electron positions. The dynamics of the electrons
requires also the gradient of the wave function with respect to electron
coordinates, and the computation of the kinetic energy needs the Laplacian
of the wave function. The computation of the gradient and Laplacian of
a Slater determinant requires the adjugate of the Slater matrix:

\begin{equation}
  \nabla_j D_i = \sum_k \left[\nabla S_i \right]_{jk} \left[\mathrm{adj}(S_i)\right]_{kj}.
\end{equation}


The quantities involved in the calculation of the wave function are the most
important kernels to be implemented in QMCkl, because they represent the main
computational bottleneck of \ac{QMC} simulations.


\section{Work done since D1.1}

\subsection{New functionalities}

\subsubsection{Implemented kernels}

The most heavily used kernels were implemented:
\begin{itemize}
  \item the \acp{AO} with their gradients and Laplacian
  \item the \acp{MO} with their gradients and Laplacian
  \item the Jastrow factor in the form implemented in CHAMP
  \item adjugate of a matrix, using Shermann-Morrison-Woodbury or
    specialized functions for small matrices
\end{itemize}    

For the computation of the non-local component of the
\ac{ECP}, it is often needed to evaluate the \acp{AO} of
\acp{MO} at extra points. For these points, only the gradients and
Laplacian are not required. We provide also functions returning only
the values for these specific cases.


\subsubsection{Improvement of design}

In the previous design, when a QMCkl function was called the computed
data was copied into the array provided by the user. We have now
introduced the possibility to directly compute the values inside the
array allocated by the user. This possibility avoids a copy, but
can be dangerous in some situations. Therefore, the in-place functions
can be called by the user, but they are suggested as dangerous
optimizations to be done when the user is sure that everything works
well with the standard version of the function.

In the initial design, the \acp{AO} and \acp{MO} were computed at the
electron positions. We realized that we could generalize the
usage of QMCkl beyond QMC, for example, for the computation of \acp{MO} on a grid.
In this context, the points no longer correspond to electrons.
Hence, we have introduced the concept of points instead of electron
positions. Electrons only start to make sense in the library for the
computation of Slater determinants, or Jastrow factors. At the
one-electron level, we stick to the concept of points.

Finally, in the initial design, the number of electrons (or points,
now) was fixed equal to the number of electrons in the system
multiplied by the number of walkers. The fact that \acp{MO} need to be
computed both at electron positions and at the extra points required
for the \ac{ECP} raised a design problem. We have redesigned
the library so that the number of points can be different from one
call to the next one to simplify the use of the library.

To simplify even more the usage of the library, we have introduced the
possibility of loading the input data from a TREXIO file in a single
call, avoiding loading all the input parameters one by one. Now, using
QMCkl has become extremely simple if the trial wave function can be
stored in a TREXIO file. Today, TREXIO is interfaced with all the
codes used in TREX and with the external codes we use, so this
modification was a significant improvement.

\subsubsection{Python interface}

We have introduced the automatic generation of a Python binding of
QMCkl using SWIG. This required the creation of \emph{safe} interfaces,
in which when a pointer to an array is passed to a function, the size
of the array is passed as well. This was quite an important change in
the API, but it also allows one to check that the arrays given by the user
are large enough to avoid writing in illegal memory domains.

\subsubsection{Portability}

Portability is an important pillar of QMCkl. Therefore, it is
continuously tested on different architectures (Linux, MacOS) with
different compilers (GNU, Intel, Nvidia, IBM). Today, all possible
combinations produce a working library.


\subsection{Documentation}
 %% TODO
% Examples.org
% Installation instructions
% Tutorials

\subsection{Building the library}

\subsubsection{Autotools}

We have entirely rewritten the build system, and we are now using
Autotools. As a testing framework is integrated in Autotools, we have
removed the dependency on munit.

The Autotools scripts are written such that out-of-source builds can
be made. This is important because it enables the possibility to have
multiple build directories configured differently: with or without the
activation of HPC-optimized functions, with different compilers, etc.
In addition, the out-of-source possibility is recommended for creating RPM
or Debian packages, so we have decided to implement this possibility
from the very beginning.

We have also implemented different building options, such as
activating or not the creation of the Python binding, or the
documentation, which can be quite expensive to build. We also have
added a simple flag to use the QMCkl-dgemm library developed in WP3 for
fast small-matrix multiplications.

\subsubsection{Packaging}

We have prepared the environment such that packaging will be
simple. As we are now using Autotools, the integration into the Spack
package manager is trivial. We have created a package for Spack, but
we are waiting for the library to be ready to be released before we
push the Spack package upstream.
Similarly, we have prepared packages for Guix and Python-pip that
will be pushed in the next release.

Many of our users are using CMake and not Autotools for their code. So
we also provide a \texttt{FindQMCKL} CMake file to help CMake detect
the presence of QMCkl. In addition, we generate files for
\texttt{pkg-config} such that linking a code with QMCkl should be as
simple as possible.

\subsection{Usage of QMCkl in external software}

We have started to use QMCkl in our software. The first use was for
the TREXIO-tools package.

TREXIO-tools is a repository of tools to interact with TREXIO files. A
common use of this tool is to check that the wave function written in
the TREXIO file is consistent. To do that, we compute numerically the
overlap matrix of the \acp{MO}, and compare it with the identity
matrix as the \acp{MO} are expected to be orthonormal. If this test
fails, it indicated to the user that the data written by the user is
inconsistent.

In the early days, this computation was performed in Python. Now, if
the QMCKl Python interface is detected on the system, this computation
is performed with the QMCkl Python binding. Orders of magnitudes were
observed in terms of performance, and the number of grid points can
now be made large enough to have an acceptable precision in the
numerical integrals.

This example demonstrates that QMCkl can be very useful in a non-QMC
framework, for performing three-dimensional visualizations of molecular
orbitals of large systems, for example, or for the computation of the
\acp{AO} on a grid for \ac{DFT} calculations. In this context, we are
initiating a collaboration with the Qubit Pharmaceuticals company
which is interested in using QMCkl in their software.

\subsubsection{Feedback from HPC experts}

There has been a strong interaction between WP1 and WP3 for the
development of high-performance versions of the library. In this
collaboration, some important feedback has been given for redesigning
the internal data-structures, to facilitate the development of
high-performance functions. 
For example, the creation of QMCkl-dgemm library induced the
abstraction of vectors, matrices and tensors in QMCkl such that normal
arrays are used in the absence of QMCkl-dgemm, but more complex data
structures can be used in HPC-optimized variants.


\subsubsection{Feedback from users}

We have started the integration of QMCkl in QMC=Chem, CHAMP and
TurboRVB. For each code, unexpected problems appeared, which caused some
adjustments in the \ac{API}. Now that we have received feedback from three
different codes, we believe that most of the use cases have been
explored and that the next users will be able to integrate QMCkl into
their code without too much effort.

\section{Conclusion}

Today, the three TREX QMC codes are calling QMCkl functions, and have
observed a significant speedup. But there is still more work to be
done to use more kernels from QMCkl, and to re-structure the codes to
make a better use of the library, in the domain where it performs best.

